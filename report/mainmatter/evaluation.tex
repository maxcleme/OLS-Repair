%Evaluation
	%Complexité
	%Performance
	%Facilité d'utilisation
	%...

\chapter{Évaluation}
	\thispagestyle{document}

\section{IntroclassJava}

Le dataset IntroclassJava\footnote{\url{https://github.com/Spirals-Team/IntroClassJava}} est un ensemble de programmes java buggés d'étudiants de L1 de l'Université de Californie. Ce dataset est une transformation de Introclass\footnote{\url{http://dijkstra.cs.virginia.edu/genprog/resources/autorepairbenchmarks//IntroClass/}}, codé en C. Chaque programme vient avec un moins un test qui ne passe pas. L'évaluation de OLS-Repair a été effectué uniquement sur les projets \textit{median} et \textit{smallest}.

\subsection{Transformation}

Dans le but de respecter le scope définie en \ref{sec:scope}, des transformations synthaxiques ont été réalisé sur le dataset. Ces transformations ne changent en rien les spécifications du programme, elles permettent cependant à OLS-Repair d'effectuer la phase de collecte correctement et de rendre le code plus réaliste. Les figures \ref{fig:dataset_avant_transfo} et  \ref{fig:dataset_apres_transfo} montre la transformation apportée. Les projets \textit{median} et \textit{smallest} transformés sont disponible sur GitHub de OLS-Repair\footnote{\url{https://github.com/maxcleme/OLS-Repair}}.



\begin{figure}
\begin{lstlisting}
public void test() throws Exception {
     median mainClass = new median();
     String expected =
          "Please enter 3 numbers separated by spaces > " + 
          "6 is the median";
     mainClass.scanner = new java.util.Scanner("2 6 8");
     mainClass.exec();
     String out = mainClass.output.replace("\n", " ").trim ();
     assertEquals(expected.replace(" ", ""), out.replace(" ", ""));
}
\end{lstlisting}
\label{fig:dataset_avant_transfo}
\caption{Exemple de test avant transformation}
\end{figure}


\begin{figure}
\begin{lstlisting}
public void test() throws Exception {
     median mainClass = new median();
     int expected =  6 ;
     int[] input = {2,8,6};
     Assert.assertEquals(expected, mainClass.median(input));
}
\end{lstlisting}
\label{fig:dataset_apres_transfo}
\caption{Exemple de tess après transformation}
\end{figure}

\subsection{Redondance}

Comme expliqué en \ref{sec:principe}, les spécifications sont indépendantes du code présent dans la méthode contenant un bug. Si OLS-Repair peut réparer une version du dataset, alors toutes les autres versions ayant les mêmes spécifications seront réparables également.

\section{Résultats}

Les constantes ajoutées à $I$ lors de la génération du problème SMT ont un rôle important pour le solver. La table \ref{table:constantes} montre que les lignes synthétisées diffèrent en fonction de ces constantes. Pour réaliser cette évaluation, uniquement les spécifications de la classe dit \textit{Whitebox} ont été prise en compte.  

\begin{table}[H]
\centering
\begin{tabular}{|l|c|c|c|c|}
\hline
\multicolumn{1}{|c|}{\multirow{2}{*}{Constante}} & \multicolumn{2}{c|}{Median} & \multicolumn{2}{c|}{Smallest} \\ \cline{2-5} 
\multicolumn{1}{|c|}{}                           & Whitebox     & Blackbox     & Whitebox      & Blackbox      \\ \hline
$\emptyset$                                           & 6/6          & 3/7          & 8/8           & 3/8           \\ \hline
{[}0{]}                                          & 6/6          & 3/7          & 8/8           & 0/8           \\ \hline
{[}-1 ; 1 {]}                                    & 6/6          & 4/7          & 8/8           & 3/8           \\ \hline
{[}-1 ; 0 ; 1{]}                                 & 6/6          & 4/7          & 8/8           & 0/8           \\ \hline
{[}2 ; 4 ; 8 ; 16 ; 32 ; 64{]}                   & 6/6          & 3/7          & 8/8           & 3/8           \\ \hline
\end{tabular}
\caption{Impactes des constantes sur la pertinence de la solution}
\label{table:constantes}
\end{table}

\par Peut importe les constantes, les spécifications de \textit{Whitebox} sont toujours respectées car elles sont définie dans le problème SMT. Par contre, les spécifications de  \textit{Blackbox} ne sont pas toujours respectées pour cause que la ligne synthétisé n'est pas assez générique. En effet, la présence de constante tel que 0 peu mener le solver vers une solution triviale, comme pour \textit{Smallest} où toutes les sorties $O$ de \textit{Whitebox} sont égales à 0.

\newpage
En généralisant sur l'ensemble du dataset IntroclassJava original, OLS-Repair peut réparer les 48 versions défaillantes de \textit{median} ainsi que les 45 versions défaillantes de \textit{smallest} car les tests sont identiques pour chaque versions. Seulement, uniquement les spécifications \textit{Whitebox} sont respectées. Comme expliqué en \ref{table:constantes}, il est possible que la ligne synthétisée respecte certaine spécification \textit{Blackbox}, cependant elles sont ne sont jamais respectées dans l'intégralité. Cependant, si l'on regarde uniquement les spécifications prise en compte, OLS-Repair répare 100\% des projets considérés.





